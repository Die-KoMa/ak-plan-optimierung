\documentclass[A4]{article}

\usepackage[ngerman]{babel}
\usepackage[utf8]{inputenc}
\usepackage[T1]{fontenc}
\usepackage{lmodern}
\usepackage{amsmath,amssymb,amsfonts,amsthm,mathtools}
\usepackage{enumerate}

\usepackage[a4paper]{geometry}

\title{Erstellung eines BuFaTa-AK-Plan als Problem der ganzzahligen lineare Optimierung}


\setlength{\parindent}{0mm}

\begin{document}
\maketitle


Das Erstellen eines AK-Plans für eine Konferenz ist ein schwieriges Problem.

Im folgenden wird fast alles außer Acht gelassen und ganze als Problem der ganzzahligen linearen Optimierung betrachtet.

Hierbei wird auf ein paar Bedürfnisse der KoMa eingegangen.

\section{Problembeschreibung}
Die Annahme ist, dass es  AK-Slots gibt und die Orga Räume für diese AK Slot zur Verfügung stehen hat.

Wir wollen die relative Überschneidungszahl pro Person minimieren,
soll heißen die Summe (oder das Maximum), der Quotienten von Überschneidungen einer Person zur Anzahl von AKs, die diese Person besuchen will soll klein sein.\footnote{Ist das dies beste Sache die man optimieren kann? Sollte man dei Summe oder das Maximum dieser Quotienten minimieren?}

Wir haben folgenden Einschränkungen:
\begin{description}
	\item[(Nicht-) Verfügbarkeit.] Nur in verfügbaren Räumen kann etwas stattfinden.
	\item[Gleichzeitigkeit.] In einem Zeitslot kann in einem verfügbaren Raum maximal ein Ak stattfinden.
	\item[Stattfinden.] Jeder Ak muss (irgendwo irgendwann genau einmal) stattfinden.
	\item[Leitung.] Eine Person kann nur einen Ak gleichzeitig leiten.
	\item[Kapazität.] Es können maximal so viele Leute in einen Raum wie er eine Raumkapazität hat.
	  \footnote{Das folgende ILP geht konservativ davon aus, dass alle die kommen wollen auch kommen. Sollte man dies abändern?}
	\item[Resos.] Man will Resos-Aks nur in speziellen Zeitslots haben (z.B. Donnertags und Freitags), daher spzifiziert man was ein Reso-AK ist und in welchen Zeitslots Reso-AKs erlaubt werden.
	\item[Blacklist.] Man kann ähnlich wie Reso-AKs für AKs Zeitslots blacklisten.
\end{description}

\section{Ganzzahliges Lineares Programm}
Formal ausgedrückt heißt:

Wir haben folgendes gegeben:
\begin{itemize}
		\item $Slot$, die Menge der Zeitslots
		\item $Room$, die Menge der Räume
		\item $Person$, die Menge der teilnehmenden Personen
		\item $AK$, die Menge der Arbeitskreise
		\item $rooms: Slot \rightarrow 2^{Room} $
					mit $rooms(s)$ die Menge der in Zeitslot $s$ verfügbaren Räumen
		\item $attend: Person \rightarrow 2^{AK} $
					mit $attend(p)$ die Menge der AKs an der Person $p$  teilnehmen will
		\item $lead: Person \rightarrow 2^{AK} $
					mit $attend(p)$ die Menge der AKs an der Person $p$ leiten will, $lead(p) \subseteq attend(p)$
		\item $cap: Room \rightarrow \mathbb{N} $ Raumkapazität
		\item $reso: AK \rightarrow \{0,1\}$ Boolesche, ob ein Ak ein Reso-Ak ist
		\item $resoable: Slot \rightarrow \{0,1\}$ Boolesche, ob in einem Zeitslot ein Reso-AK stattfinden darf
		\item $blacklist: AK \rightarrow 2^{Slot}$ mit $blacklist(a)$ Menge der Zeitslots in denen Ak $a$ nicht stattfinden darf.
\end{itemize}

In dem ILP gibt es die booleschen Variablen $x_{s,r,a}$ ob Ak $a$ in Zeitslot $s$ in Raum $r$ stattfindet
und die ganzzahligen Variablen $y_{s,p}$ wie viele Überschneidungen eine Person $p$ in Slot $s$ hat.

\newgeometry{left=0.5cm}
Das IPL ist:
\begin{alignat*}{4}
	&&&&\llap{$\min \sum\limits_{p\in Person} \frac{1}{|attend(p)|} \sum\limits_{s\in Slot} y_{s,p}$} \\
	&\text{subject to} \\
	&& x &&& \in \{0,1\}^{ available   \times A}  && \\
	(&\forall s\in Slot\ \forall r \in Room\ \forall a \in AK  & x_{s,r,a} &&& \in \{0,1\}  && \text{Boolesche}) \\
	(&\forall s\in Slot\ \forall r \notin rooms(s)\ \forall a \in AK  & x_{s,r,a} &&&= 0 && \text{Nichtverfügbarkeit}) \\
	&\forall s\in Slot\ \forall r \in rooms(s) & \sum_{a \in AK} x_{s,r,a} &&&\leq 1 && \text{Gleichzeitigkeit} \\
	&\forall a \in AK & \sum_{s\in Slot} \sum_{r \in rooms(s)} x_{s,r,a} &&&= 1 && \text{Stattfinden} \\
	&\forall p\in Person\ \forall s \in Slot & \sum_{r \in rooms(s)} \sum_{a \in lead(p)} x_{s,r,a} &&&\leq 1 && \text{Leitung} \\
	&\forall s\in Slot\ \forall r \in rooms(s) & \sum_{p \in Person} \sum_{a \in attend(p)} x_{s,r,a} &&&\leq cap(r) && \text{Kapazität} \\
	&\forall a \in AK\  \forall s\in Slot & \sum_{r \in rooms(s)} x_{s,r,a} &&&\leq (resoable(s) \vee\neg reso(a)  ) \ && \text{Reso} \\
	&\forall a \in AK\  \forall s\in blacklist(a)\ \forall r \in rooms(s) & x_{s,r,a} &&&=0  && \text{Blacklist} \\
	&\forall s \in Slot\ \forall p\in Person  & \sum_{r \in rooms(s)} \sum_{a \in attend(p)} x_{s,r,a} & - & y_{s,p} &\leq 1 && \text{Überschneidungs-Variable} \\
	&\forall s \in Slot\ \forall p\in Person  &&&  y_{s,p} &\geq 0 && \text{nicht negativ}
\end{alignat*}

mit $available=\{ (s,r) : s \in Slot, r \in rooms(s) \}$.

\restoregeometry 


\end{document}